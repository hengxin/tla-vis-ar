% intro.tex

%%%%%%%%%%%%%%%%%%%%%%%%%%%%%%
\section{Introduction} \label{section:intro}

\bfit{Motivations.}
Distributed consistency models are quite tricky to understand.
Even worse, in the literature, there are often several variants of a specific consistency model.
For example, Jiang et al. presents six variants of causal consistency in~\cite{SpecFramework:SRDS2020} for non-transactional databases.
Crooks presents a hierarchy of eight variants of snapshot isolation~\cite{Crooks:PODC2017} for transactional databases.
It is difficult, even for experts,
\begin{description}
  \item[\taskchecking:] to check whether a given history satisfies some consistency model or not;
  \item[\taskgenerating:] to come up with histories under some constraints (e.g., on size)
    that satisfy or refuse some consistency model; and
  \item[\taskcomparing:] to tell the differences between two consistency models
    by presenting distinguishing histories that satisfy one consistency model but refuse the other.
\end{description}
All the three tasks are concerned about histories, which are easier for human to understand.

\red{programs vs. litmus tests vs. histories vs. abstract executions}

\red{TODO: 1) how to control the execution of a given history; 
2) random tests vs litmus tests;
3) how to divide histories into equivalent classes (reduce reduction)}

\bfit{Challenges.}
\begin{itemize}
  \item How to formally express consistency models?
  \item generating too many (redundant) histories
\end{itemize}

\bfit{Our Contributions.}
\begin{itemize}
  \item Alloy$\ast$ model of both non-transactional and transactional consistency models
  \item Two case studies
\end{itemize}
%%%%%%%%%%%%%%%%%%%%%%%%%%%%%%
