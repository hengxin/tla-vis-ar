% si-variants.tex

%%%%%%%%%%%%%%%%%%%%%%%%%
\subsection{Snapshot Isolation Variants} \label{ss:si-variants}

%%%%%%%%%%%%%%%%%%%%
\subsubsection{The \alloystar{} Model} \label{sss:si-alloystar}

We first define the $(vis,ar)$ framework using \alloy.
At the top of the model, several type signatures (``sigs") are defined.
According to Definition~\ref{}, a history is a set of transactions and
a transaction is a sequence of several events.
An event contains an identifier \id{} and an operation $\op$.
Since we focus on read/write key-value stores in this paper,
there are two kinds of operations, i.e., \Read{} and \Write.
The \Read{} and \Write{} operations reads from
and writes to a key of type \Key.
Additionally, the signature $T$ defines \po{} relation that
totally orders all the events in $T$.
An Abstract execution are an enriched history with two relations, \vis{} and \ar{},
which are relations over transactions.

Below the sig definitions are the basic constraints (``facts").
For example, \vis{} is a an acyclic relation 
and \ar{} is a total order over all transactions in a given history.
Any valid instance of the snapshot isolation variants specified in the $(vis,ar)$ framework
is required to satisfy all of the listed facts to be considered well-formed,
regardless of whether it is considered valid according to the variant.

Since an transactional consistency model is a set of consistency axioms, 
we first describe a set of consistency axioms 
including \internal{}, \ext{}, \noconflict{}, \prefix{}, \transvis{} and \totalvis.
Each consistency axiom is a predicate,
which specifies the properties of $vis$, $ar$ that must hold true for
an execution which satisfies the axiom.
Based on these consistency axioms,
we formalize the transactional consistency models in Table~\ref{}.

%%%%%%%%%%%%%%%%%%%%
\subsubsection{\taskchecking} \label{sss:si-taskchecking}

%%%%%%%%%%%%%%%%%%%%
\subsubsection{\taskgenerating} \label{sss:si-taskgenerating}

%%%%%%%%%%%%%%%%%%%%
\subsubsection{\taskcomparing} \label{sss:si-taskcomparing}

%%%%%%%%%%%%%%%%%%%%%%%%%
