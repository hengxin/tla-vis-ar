% cc-variants.tex

%%%%%%%%%%%%%%%%%%%%%%%%%
\subsection{Causal Consistency Variants} \label{ss:cc-variants}

%%%%%%%%%%%%%%%%%%%%
\subsubsection{The \alloystar{} Model} \label{sss:cc-alloystar}

We first define the $(vis,ar,V)$ framework using \alloy.
At the top of the model, several type signatures (``sigs") are defined.
According to Definition~\ref{}, an event $E$ contains
an operation \op{} and a return value \rval.
Additionally, the signature $E$ defines \so relation that
originates form itself to other events on the same session,
\session{} relation that indicates where $E$ performs on
and \ve{} relation that depicts the events in $V(e)$ (Definition~\ref{}).
Since we focus on read/write key-value stores in this paper,
there are two kinds of operations, i.e., \Read{} and \Write.
The \Read{} and \Write{} operations reads from
and writes to a key of type \Key.

Below the sig definitions are the basic constraints (``facts").
For example, \so{} is a total order over all events on the same session.
Any valid instance of the model specified in the $(vis,ar,V)$ framework
is required to satisfy all of the listed facts to be considered well-formed,
regardless of whether it is considered valid according to the model.
Then we describe the return value consistency $\retval(\intreg)$ (Definition~\ref{})
by the predicates ``\texttt{ReadLastVisibleWrite}" and ``\texttt{VeIsReasonable}".

Based on the $(vis,ar,V)$ framework in Figure~\ref{},
we formalize the recipes for $vis$, $ar$ and $V$ in Table~\ref{} and
six causal consistency variants specified in the $(vis,ar,V)$ framework.
The predicate for each causal consistency variant specifies
the properties of $vis$, $ar$ and $V$ that must hold true for
an execution which satisfies the model.

%%%%%%%%%%%%%%%%%%%%
\subsubsection{\taskchecking} \label{sss:cc-taskchecking}

%%%%%%%%%%%%%%%%%%%%
\subsubsection{\taskgenerating} \label{sss:cc-taskgenerating}

%%%%%%%%%%%%%%%%%%%%
\subsubsection{\taskcomparing} \label{sss:cc-taskcomparing}

%%%%%%%%%%%%%%%%%%%%%%%%%%%%%%
